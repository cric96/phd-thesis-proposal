\documentclass[11pt]{article}
\usepackage{mgates-letter}
\definecolor{dark_blue} {rgb}{0., 0., 0.65}
\usepackage{makecell}

\usepackage{textcomp}
\usepackage{mathrsfs}  % mathscr font
\usepackage{boxedminipage}
\usepackage{rotating}
\usepackage{csquotes}
%\usepackage{natbib}
\usepackage[colorlinks, filecolor=dark_blue, urlcolor=dark_blue, linkcolor=black, citecolor=black]{hyperref}

\begin{document}
\sloppy
\begin{center}
	{{
		\Large{
			\textsc{PhD Programme in Computer Science and Engineering \\ 
			\vspace{4mm}
			Cycle XXXVI}
			}
	}} 
	\rule[0.1cm]{\textwidth}{0.1mm}
	\rule[0.4cm]{\textwidth}{0.6mm}
\end{center}

\begin{center}
	{\LARGE{Engineering Cyber-Physical Swarms with Aggregate Computing}} \\
	\vspace{4mm}
	{\large{PhD Thesis proposal}} 
	\vspace{4mm}
\end{center}
\vspace{8mm}
\par
\noindent
\begin{minipage}[t]{0.47\textwidth}

{\large{Commission: \\\bf
Prof. Mirko Viroli \\
Prof. Andrea Omicini \\
Prof. Matteo Ferrara} 
}
\end{minipage}
\hfill
\begin{minipage}[t]{0.47\textwidth}
	\raggedleft
	{
		\large{PhD Student: \\\bf Gianluca Aguzzi}
	}
\end{minipage}
\vspace{10mm}

{
	\raggedright
	\rule[0.1cm]{\textwidth}{0.6mm}
	\rule[0.5cm]{\textwidth}{0.1mm}
}

\newcommand{\rev}[1]{{
	%\color{red}
	#1
	}}
\abstract{
This thesis aims to define a path towards the engineering of Cypher-Physical Swarms (CPSWs) - where swarms are seen in a modern key, including swarm robotics, large-scale IoT scenarios, and crowd engineering (tracking and control). 
%
CPSWs are backed by emerging trends of autonomic, pervasive, ubiquitous computing that foster a vision of distributed systems composed of a considerable number of simple entities that collectively perform complex tasks.
%
These systems are similar to natural social-animal groups, where plentiful animals (like ants, sheep, ...) achieve complex tasks (foragging, flocking, ...).
%
Furthermore, these behaviour are typically achieved through self-organization, making the system robust to failures and highly scalable.
%
Inspired by nature, we define the notion of ``Cypher-Physical Swarm'' -- the extension of `swarms' (of people, of computers, of things) in computer science.
Engineering this kind of system with traditional approaches is inadequate, 
due to the distributed control, high-rate failure, openness, and local-to-global behaviour mapping.
%
Therefore, my thesis is focused on finding a systematic and reproducible way to design CPSWs. 
In particular, we plan to leverage Aggregate Computing -- a novel top-down global-to-local programming model.
The latter is used to \emph{program} self-stabilization behaviours through a functional and composable manipulation of \emph{computational fields}.
%
At the same time, we consider leveraging Machine Learning to improve \emph{adapativity} -- both at the middleware level and the Aggregate Computing specification language.
%
As a final result of my PhD, we intend to reach a seamless integration of Aggregate Computing and Machine Learning, making the collective behaviour specification smarter, more effective and faster
} %% TODO add discussion on middlewares?? 

\tableofcontents
\section{Introduction}

\section{Background}
\subsection{Swarm intellingence}
\subsection{Multi-Agent systems}
\subsection{Aggregate Computing}
\section{Contributions}

\subsection{Preliminary contributions}
\subsubsection{ScaFi-Web -- A tools for a distributed monitoring}
\subsubsection{ScaFi Loci -- Towards a type-safe deployment of Pulverised Architecture}
\subsubsection{ScaFi-Web -- A tools for a distributed monitoring}
\subsection{Future works}
\subsubsection{Swarm API}
\subsubsection{Middleware-level learning}
\section{Conclusion}
\nocite{*}
\bibliographystyle{ieeetr}
\bibliography{biblio}

\end{document}
